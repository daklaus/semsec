\documentclass[]{beamer} % die option 't' sorgt dafür, dass der content nicht automatisch auf die folie zentriert

\usepackage[utf8]{inputenc}
% \inputencoding{latin1} % for windows user
\usepackage[T1]{fontenc}
\usepackage{lmodern}
\usepackage{default}
\usepackage[english]{babel}
\usepackage{graphicx}
\graphicspath{{figures/}{imgs/}}
\usepackage[backend=bibtex,style=ieee]{biblatex}
\bibliography{bib/db.bib}
\usepackage[acronym,shortcuts]{glossaries}
\makeglossaries
\InputIfFileExists{abbrev/acronyms}{}{}


\usetheme{INSO}

\title{Title of the presentation}
%\subtitle{Short title}
\author{<Firstname> <Lastname>}
\matrnr{<matrnr>}
\date{\today}
%\advisor{%
%	Betreuer 1\\
%	Betreuer 2
%}

\setlength{\parskip}{1em}

\begin{document}

\maketitle

\begin{frame}{Outline}
	\begin{minipage}[t][10em][t]{\linewidth}
		\tableofcontents
	\end{minipage}
\end{frame}

\section{Introduction}

\begin{frame}{Introduction}
\end{frame}

\section{Related Work}

\begin{frame}{Related Work}
\end{frame}

\section{\protect\dots}

\begin{frame}{\dots}
\end{frame}

\section{Conclusion}

\begin{frame}{Conclusion}
\end{frame}

\begin{titleframe}
	\begin{center}
	\alert{\Large Thank you!}
	\end{center}
\end{titleframe}

\end{document}
